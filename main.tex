\documentclass{report}
\usepackage[utf8]{inputenc}
\usepackage{geometry}
 \geometry{
 a4paper,
 total={150mm,230mm},
 left=25mm,
 top=20mm,
 }

\begin{document}
\large{Roll NO-ME21B050}
\newline

\huge{\textit{UNIVERSAL LAW OF GRAVITATION}}
 
\section{\{LAW OF GRAVITATION}}
\Large{
\normalsizeThe Newton’s law of gravitation, statement that any particle of matter in the universe attracts any other with a force varying directly as the product of the masses and inversely as the square of the distance between them
\newline
\newline

\begin{document}
\huge{\textit{{ UNDERSTANDING OF GRAVITATION  LAW}}}
 
\section{\underline{OVERVIEW:-}}
\Large{
\normalsizeThe Isaac Newton put forward the law in 1687 and used it to explain the observed motions of the planets and their moons, which had been reduced to mathematical form by Johannes Kepler early in the 17th century.
\newline
\newline
\textbf{\textit{DEFINATION:}} \normalsize the law states that every point mass attracts every other point mass by a force acting along the line intersecting the two points. The force is proportional to the product of the two masses, and inversely proportional to the square of the distance between them
}
\section{\{Formula:-}}
 
\large{$F= G(m_1*m_2)/r^2$}
\huge{ 
\begin{center}
\begin{tabular}{ |c|c| } 
 \hline
 $G$ & Universal gravitational constant \\
 $r$ & Distance between $m_1$ and $m_2$ \\
 $m_1$ & Mass  of particle 1 \\
 $m_2$ & Mass of particle 2 \\
 F & Force between the two particles \\
 \hline
\end{tabular}
\end{center}
}
